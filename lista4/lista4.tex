\documentclass[a4paper, 11pt]{article}
\usepackage[utf8]{inputenc}
\usepackage[T1]{fontenc}

\usepackage[margin=1in]{geometry}
\usepackage{amsmath, amsfonts, amsthm, amssymb, amsxtra}

\usepackage{graphicx}
\usepackage{float}

\usepackage[dvipsnames]{xcolor}

\usepackage{tabularray}
\usepackage{enumitem}
\usepackage{multicol}

\usepackage{notomath}

\usepackage{hyperref}
\hypersetup{hidelinks}

\usepackage[portuguese]{babel}

\usepackage{tikz}
\usetikzlibrary{intersections, angles, calc, positioning}
\usetikzlibrary{shapes.geometric, arrows.meta}
\usetikzlibrary{decorations.pathmorphing, decorations.pathreplacing}

\setlength{\parindent}{0pt}
\setlength{\parskip}{5pt}

\makeatletter
\newcommand{\type}[1]{\def\@type{#1}}

\renewcommand*{\maketitle}{%
\vspace{-0.5cm}
\begin{tikzpicture}[remember picture, overlay]
    \node[anchor=south, align=center] (date) at ($(current page.north) + (0,-110pt)$) {\@date};
    \node[anchor=south, align=center, font=\itshape] (author) at (date.north) {\@author};
    \node[above=10pt of author, align=center, font=\scshape] (type) {\@type};
    \node[anchor=south, align=center, font=\bfseries\large] (title) at (type.north) {\@title};
    \node[anchor=west] (logo) at ($(current page.north west) + (\Gm@lmargin, -70pt)$) {\includegraphics[height=3.2cm]{../ufs_vertical_positiva.eps}};
    \draw ($(current page.north west) + (\Gm@lmargin, -120pt)$) -- ($(current page.north east) + (-\Gm@lmargin, -120pt)$);
\end{tikzpicture}
\vspace{45pt}
}%
\makeatother

\newcommand{\cT}{\mathcal T}
\newcommand{\bC}{\mathbf C}

\title{Tópicos de Geometria e\\Topologia}
\type{Lista 4}
\author{Bruno Sant'Anna}
\date{19 de fevereiro de 2024}

\begin{document}
\maketitle

Sejam $X, Y, Z$ espaços topológicos conexos e localmente conexos por caminhos.
Considere mapas contínuos $p : X \to Y$, $q : Y \to Z$ e $X \to Z$ tais que $r = q \circ p$ e assuma que $p$ e $r$ são mapas de recobrimento.
Mostre que $q$ é um mapa de recobrimento.

\textcolor{black!30}{\hrulefill}

De fato, $q : Y \to Z$ é um mapa de recobrimento.
Como $r : X \to Z$ é um mapa de recobrimento, então, dado $z \in Z$ existe uma vizinhança $\Gamma$ de $z$ tal que $r^{-1}(\Gamma) = \sqcup\, \Omega_\alpha$. 
Da definição de $r$ segue que 
\[
    (q \circ p)^{-1}(\Gamma) = p^{-1} \circ q^{-1} (\Gamma) = {\textstyle\bigsqcup} \Omega_\alpha
\]
como $p$ é mapa de recobrimento, em particular $p$ é sobrejetívo, consequentemente tem inversa à esquerda, daí
\begin{align*}
    p \circ p^{-1} \circ q^{-1} (\Gamma) &= p\left( {\textstyle\bigsqcup}  \Omega_\alpha \right)\\
    q^{-1}(\Gamma) &= {\textstyle\bigsqcup}  p(\Omega_\alpha)
\end{align*}
onde $p(\Omega_\alpha)$ é sempre aberto pois mapas de recobrimentos levam conjuntos abertos em abertos.
Ou seja, dado um $z \in Z$ existe uma vizinhança de $z$ tal que a imagem inversa por $q$ é a união disjunta de abertos em $Y$

Além disso precisamos mostrar que $q |_{p(\Omega_\alpha)} : p(\Omega_\alpha) \to \Gamma$ é um homeomorfismo. Com efeito, como $r$ é mapa de recobrimento, sabemos que $r|_{\Omega_\alpha} : \Omega_\alpha \to \Gamma$ é um homeomorfismo, e $r = q \circ p$, então
\begin{align*}
    r|_{\Omega_\alpha} &= (q \circ p)|_{\Omega_\alpha}\\
    &= q|_{p(\Omega_\alpha)} \circ p|_{\Omega_\alpha}
\end{align*}
onde $p|_{\Omega_\alpha}$ é um homeomorfismo pois $p$ é mapa de recobrimento, então tem inversa $p|^{-1}_{\Omega_\alpha}$, daí
\[
    q|_{p(\Omega_\alpha)} = r|_{\Omega_\alpha} \circ p|^{-1}_{\Omega_\alpha}
\]
Logo, $q|_{p(\Omega_\alpha)}$ é uma composição de homeomorfismos, que é um homeomorfismo.

Portanto, $q : Y \to Z$ é um mapa de recobrimento.

\end{document}