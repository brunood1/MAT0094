\documentclass[a4paper, 11pt]{article}
\usepackage[utf8]{inputenc}
\usepackage[T1]{fontenc}

\usepackage[margin=1in]{geometry}
\usepackage{amsmath, amsfonts, amsthm, amssymb, amsxtra}

\usepackage{graphicx}
\usepackage{float}

\usepackage[dvipsnames]{xcolor}

\usepackage{tabularray}
\usepackage{enumitem}
\usepackage{multicol}

\usepackage{notomath}

\usepackage{hyperref}
\hypersetup{hidelinks}

\usepackage[portuguese]{babel}

\usepackage{tikz}
\usetikzlibrary{intersections, angles, calc, positioning}
\usetikzlibrary{shapes.geometric, arrows.meta}
\usetikzlibrary{decorations.pathmorphing, decorations.pathreplacing}

\setlength{\parindent}{0pt}
\setlength{\parskip}{5pt}

\makeatletter
\newcommand{\type}[1]{\def\@type{#1}}

\renewcommand*{\maketitle}{%
\vspace{-0.5cm}
\begin{tikzpicture}[remember picture, overlay]
    \node[anchor=south, align=center] (date) at ($(current page.north) + (0,-110pt)$) {\@date};
    \node[anchor=south, align=center, font=\itshape] (author) at (date.north) {\@author};
    \node[above=10pt of author, align=center, font=\scshape] (type) {\@type};
    \node[anchor=south, align=center, font=\bfseries\large] (title) at (type.north) {\@title};
    \node[anchor=west] (logo) at ($(current page.north west) + (\Gm@lmargin, -70pt)$) {\includegraphics[height=3.2cm]{../ufs_vertical_positiva.eps}};
    \draw ($(current page.north west) + (\Gm@lmargin, -120pt)$) -- ($(current page.north east) + (-\Gm@lmargin, -120pt)$);
\end{tikzpicture}
\vspace{45pt}
}%
\makeatother

\newcommand{\cT}{\mathcal T}
\newcommand{\bC}{\mathbf C}
\newcommand{\bR}{\mathbf R}
\newcommand{\bS}{\mathbf S}
\newcommand{\bx}{\mathbf x}

\title{Tópicos de Geometria e\\Topologia}
\type{Lista 5}
\author{Bruno Sant'Anna}
\date{02 de março de 2024}

\begin{document}
\maketitle

Seja $Q$ uma forma quadrática positiva definida em $\bR^n$. Mostre que $Q^{-1}(y)$ é difeomorfo à esfera $\bS^{n-1}$ para todo $y > 0$

\textcolor{black!30}{\hrulefill}

Dado uma forma quadrática $Q$ temos que
\[
    Q(\bx) = \left\langle \bx, A\bx \right\rangle = \bx^T A \bx
\]
onde $A$ é uma matriz simétrica e positiva definida.
Para definir o difeomorfismo, vale lembrar que qualquer matriz simétrica pode ser transformada em uma matriz diagonal, pela relação $\Lambda = P^T A P$, onde $\Lambda$ é a matriz diagonal de autovalores de $A$, e $P$ é uma matriz inversível.

Dado $\bx \in Q^{-1}(y)$, isto é, $\bx^T A \bx = y$, defina o mapa
\[
    \begin{tblr}{
        colspec={rrcl}, 
        leftsep=1pt, rightsep=1pt
        }
        \psi: & Q^{-1}(y) & \to     & Q^{-1}(y)\\
              & \bx       & \mapsto & P \bx
    \end{tblr}
\]
onde $P$ é a matriz que diagonaliza $A$

Por outro lado, defina o mapa
\[
    \begin{tblr}{
        colspec={rrcl}, 
        leftsep=1pt, rightsep=1pt
        }
        \phi: & Q^{-1}(y) & \to     & Q^{-1}(y)\\
              & \bx       & \mapsto & S \bx
    \end{tblr}
\]
com $S$ sendo
\[
    S = 
    \begin{pmatrix}
        \sqrt{\gamma_1} & 0 & \cdots & 0\\
        0 & \sqrt{\gamma_2} & \cdots & 0\\
        \vdots & \vdots & \ddots & \vdots\\
        0 & 0 & \cdots & \sqrt{\gamma_n}
    \end{pmatrix}
\]
onde $\gamma_i = y \lambda_i^{-1}$ e $\lambda_i = \Lambda_{ii}$ é o i-ésimo autovalor.
Note que $S$ está bem definida, já que $A$ é uma matriz simétrica e positiva definida, então, todos seus autovalores são positivos, ou seja, possuem inverso multiplicativo e raiz quadrada.

Compondo os dois mapas, temos
\[
    \begin{tblr}{
        colspec={rrcl}, 
        leftsep=1pt, rightsep=1pt
        }
        \phi \circ \psi : & Q^{-1}(y) & \to     & Q^{-1}(y)\\
                          & \bx       & \mapsto & P S \bx
    \end{tblr}
\]
Perceba que $\phi \circ \psi \left(Q^{-1}(y)\right) = \bS^{n-1}$.
Com efeito, seja $\bx \in Q^{-1}(y)$, fazendo $\phi \circ \psi (\bx)$,
\begin{align*}
    (P S \bx)^T A (P S \bx) &= y\\
    \bx^T S^T (P^T A P) S \bx &= y\\
    \bx^T (S^T \Lambda S) \bx &= y\\
    \bx^T y I_n \bx &= y\\
    \bx^T \bx &= 1
\end{align*}
Pela outra definição de forma quadrática, segue que $\left\langle \bx, \bx \right\rangle = \left\lVert \bx \right\rVert^2 = 1$, ou seja $\phi \circ \psi (\bx) \in \bS^{n-1}$

Por fim, é fácil ver que $\widetilde{\phi \circ \psi} : Q^{-1}(y) \to \bS^{n-1}$ é um difeomorfismo, pois $\widetilde{\phi \circ \psi}$ é uma transformação linear, que em particular é um difeomorfismo.
\end{document}