\documentclass[a4paper, 10pt]{article}
\usepackage[utf8]{inputenc}
\usepackage[T1]{fontenc}

\usepackage[margin=1in]{geometry}
\usepackage{amsmath, amsfonts, amsthm, amssymb, amsxtra}

\usepackage{graphicx}
\usepackage{float}

\usepackage[dvipsnames]{xcolor}

\usepackage{tabularray}
\usepackage{enumitem}
\usepackage{multicol}

\usepackage{notomath}
% \usepackage{fontspec}
% \usepackage{unicode-math}
% \setmainfont{FiraGO}
% \setmathfont{Fira Math Book}

\usepackage{hyperref}
\hypersetup{hidelinks}

\usepackage{tikz}
\usetikzlibrary{intersections, angles, calc, positioning}
\usetikzlibrary{shapes.geometric, arrows.meta}
\usetikzlibrary{decorations.pathmorphing, decorations.pathreplacing}

\setlength{\parindent}{0pt}
\setlength{\parskip}{5pt}

\makeatletter
\newcommand{\type}[1]{\def\@type{#1}}

\renewcommand*{\maketitle}{%
\vspace{-0.5cm}
\begin{tikzpicture}[remember picture, overlay]
    \node[anchor=south, align=center] (date) at ($(current page.north) + (0,-110pt)$) {\@date};
    \node[anchor=south, align=center, font=\itshape] (author) at (date.north) {\@author};
    \node[above=10pt of author, align=center, font=\scshape] (type) {\@type};
    \node[anchor=south, align=center, font=\bfseries\large] (title) at (type.north) {\@title};
    \node[anchor=west] (logo) at ($(current page.north west) + (\Gm@lmargin, -70pt)$) {\includegraphics[height=3.2cm]{../ufs_vertical_positiva.eps}};
    \draw ($(current page.north west) + (\Gm@lmargin, -120pt)$) -- ($(current page.north east) + (-\Gm@lmargin, -120pt)$);
\end{tikzpicture}
\vspace{45pt}
}%
\makeatother

\newcommand{\cT}{\mathcal T}

\title{Tópicos de Geometria e\\Topologia}
\type{Lista 2}
\author{Bruno Sant'Anna}
\date{14 de janeiro de 2024}

\begin{document}
\maketitle

Sejam $X$, $Y$, $Z$ espaços topológicos, $f: X \to Y$, $g: Y \to Z$ funções contínuas e $p \in X$. 
Mostre que
\begin{enumerate}[label=\alph*., leftmargin=*]
    \item O homomorfismo induzido $(g \circ f)_* : \pi_1 (X, p) \to \pi_1 (Z, (g \circ f)(p))$ coincide com a composição dos respectivos homomorfismos induzidos $g_* \circ f_*$, onde $f_* : \pi_1 (X, p) \to \pi_1(Y, f(p))$ e $g_* : \pi_1 (Y, f(p)) \to \pi_1 (Z, (g \circ f)(p))$.
    
    \item Se $I_X$ denota a identidade de $X$, então o homomorfismo induzido $(I_X)_*$ coincide com o homomorfismo identidade do grupo fundamental $\pi_1 (X,p)$.
\end{enumerate}

\textcolor{black!30}{\hrulefill}

\begin{enumerate}[label=\alph*., leftmargin=*]
    \item Se $\psi : X \to Y$ é um mapa contínuo e $p \in X$, então o mapa 
    \[
    \begin{tblr}{
        colspec={rrcl}, 
        leftsep=1pt, rightsep=1pt
        }
        \psi_* : & \,\pi_1 (X, p) &\to     & \pi_1 (X, \psi(p))\\
                 &  [\gamma]      &\mapsto & [\psi \circ \gamma]
    \end{tblr}
    \]
    é dito mapa induzido de $\psi$ e está bem definido

    Sejam $f : X \to Y$ e $g : Y \to Z$ mapas contínuos entre espaços topologicos e $p \in X$.
    Pela definição de mapa induzido temos
    \[(g \circ f)_* [\gamma] = [(g \circ f) \circ \gamma]\]
    como a composição de mapas é associativa
    \[[(g \circ f) \circ \gamma] = [g \circ (f \circ \gamma)]\]
    e novamente utilizando a definiçao de mapa induzido
    \[[g \circ (f \circ \gamma)] = g_* [f \circ \gamma] = (g_* \circ f_*) [\gamma].\]
    Então de fato, o homomorfismo induzido $(g \circ f)_*$ coincide com a composição $g_* \circ f_*$ 

    \item Se $I_X : X \to X$ é a identidade de $X$, então o mapa induzido $(I_X)_* : \pi_1 (X, p) \to \pi_1 (X, p)$ é a identidade de $\pi_1 (X,p)$. Com efeito
    \[(I_X)_* [\gamma] = [I_X \circ \gamma] = [\gamma]\]
    ou seja, a classe de $\gamma$ por meio de $(I_X)_*$ foi levada a ela mesma. Portanto $(I_X)_*$ é a identidade em $\pi_1 (X,p)$
\end{enumerate}

% \newpage

% \maketitle

% Seja $X$ um espaço topológico conexo por caminhos e considere $p \in X$. Se $e_p : [0,1] \to X$ denota o caminho constante baseado em $p$. 
% Mostre que
% \begin{enumerate}[label=\alph*., leftmargin=*]
%     \item Dois laços $f,g \in \Omega(X, p)$ são caminhos homotópicos se, e somente se $f^{-1} \cdot g$ é homotópico a $e_p$

%     \item $X$ é simplesmente conexo se, e somente se dois caminhos em $X$ com os mesmos pontos inicial e final são caminhos homotópicos.
% \end{enumerate}

% \textcolor{black!30}{\hrulefill}

% \begin{enumerate}[label=\alph*., leftmargin=*]
%     \item Dados $f,g \in \Omega(X,p)$, $f$ e $g$ são laços homotópicos se, e somente se as suas classes no grupo fundamental $\pi_1 (X,p)$ são iguais.
%     Então, 
%     \begin{align*}
%         f \simeq g &\iff [f] = [g]\\
%         &\iff [e_p] = [f^{-1}] \cdot [g]\\
%         &\iff e_p \simeq f^{-1} \cdot g
%     \end{align*}
% \end{enumerate}



\end{document}